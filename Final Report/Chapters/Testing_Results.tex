%% ----------------------------------------------------------------
%%? 1st Draft Completed
%% ----------------------------------------------------------------
\chapter{Testing and Results (1900 Words)}
\section{Data Collection}
As the project was aiming to initially understand how people manage and use their digital music collections, then to see how these could be transformed using new concepts implemented with software, a more open approach was required.

Students from the University of Southampton with an active Spotify Account were invited to take part in a participant study after development had finished. This study was performed and complied with the ethics standards set out in \lstinline|ERGO78677.A2|.

The study consisted of 1-on-1 ethnographic semi-structured interviews, typically lasting 45-60 minutes. The process was as follows:\begin{itemize}
    \item Provide the participant with an information sheet and consent form
    \item They were then asked how they build song queues - from fully random to fully ordered
    \item The following questions were asked:\begin{itemize}
        \item How is your collection organised?
        \item How do you maintain your collection?
        \item How do you create queues to listen to? (expanding on their previous answer of random vs ordered)
        \item How do you interact with your listening queues after creation (excluding the listening aspect itself)?
    \end{itemize}
    \item Then the participant loaded a playlist into the software application:\begin{itemize}
        \item Export a chosen playlist as a \lstinline|.csv| file from the Exportify web page
        \item \item Open the Audyssey application with the aformentioned \lstinline|.csv| file
    \end{itemize}
    \item Then the user was walked through the application and then they gave their thoughts on the application and concepts within it.
\end{itemize}

Due to the semi-structured nature, some questions and topics were explored to different degrees depending on the different answers and behaviours of the participants. During the interviews, notes were taken to be used in an inductive coding process.

This process was done iteratively, modifying and combing codes after each interview. Then these codes were grouped to form a hierarchy of themes and sub-themes, along with the occurrence count for each code. This process follows from the Thematic Analysis process set out by \cite{
    %todo Joy, Braun, Clarke's paper on reflexive thematic analysis
}

%% ----------------------------------------------------------------
\section{(TODO Add quotes + who said what)Thematic Analysis of Interview Pre-application}
\begin{longtable}[c]{| c | c | c | c}
    \caption{Hierarchical Table of Themes with Counts.} \\
    \toprule
    Theme & Sub-Theme & Code & Count\\
    \midrule
    \endfirsthead
    
    %\caption[]{Hierarchical Table of Themes, Sub-Themes, Codes and Counts.}
    \toprule
    Theme & Sub-Theme & Code & Count\\
    \midrule
    \endhead

    \multirow{9}{6em}{
        Collection Organisation
    } & \multirow{2}{*}{Method} & Singular Playlist & 2\\* \cmidrule{3-4}
            & & Multiple Playlists & 4 \\*
        \cmidrule{2-4}
        & \multirow{2}{*}{Full Collection Understanding} & Vague & 1\\*  
            \cmidrule{3-4}
            & & Strong & 5 \\*
        \cmidrule{2-4}
        & Mental Model & Distinctly Unique & 2 \\*
        & Familiar Structure & Easy to Use & 4 \\*% all the playlist peeps
        & Dark Spots & Forgotten/Unfamiliar Songs & 3\\*
        \cmidrule{2-4}
        & \multirow{2}{*}{Growing Song Count} & Always Adding & 6\\*
            \cmidrule{3-4}
            & & Doesn't Remove Songs & 5 \\*
    \midrule

    \multirow{5}{6em}{
        Playlist Management
    } & \multirow{5}{*}{Unique Identity} & Per Artist & 1 \\*
        \cmidrule{3-4}
            & & Per Time Period & 2 \\*
            \cmidrule{3-4}
            & & Per Genre & 2 \\*
            \cmidrule{3-4}
            & & Per Mood/Vibe & 3 \\*
            \cmidrule{3-4}
            & & Activity & 2 \\*
        \cmidrule{2-4}
        & \multirow{4}{*}{Friction/High Mental Load} & Creating DesiredPlaylists & 2\\*
        \cmidrule{3-4}
            & & Switching between Playlists & 2\\*%* for creating queues
            \cmidrule{3-4}
            & & Infrequent Playlist Creation & 3\\*
            \cmidrule{3-4}
            & & Cleaning Playlist Contents & 1\\*
        \cmidrule{2-4}
        & \multirow{2}{*}{Absolute/Fixed Ordering} & Playlist Contents &3\\*
        \cmidrule{3-4}
            & & Specific Audio Experiences & 3\\*
    \midrule
    
    \multirow{12}{6em}{Listening Journeys (Queue)} & \multirow{2}{*}{Creation Process} & Passive: Shuffle & 5\\*
    \cmidrule{3-4}
            & & Active: Manually Create Queue & 2\\*
        \cmidrule{2-4}
        & \multirow{3}{*}{Source} & Individual Playlist & 4\\*
        \cmidrule{3-4}
            & & Full Collection & 4\\*
            \cmidrule{3-4}
            & & Affected by Recency Bias & 3\\*
        \cmidrule{2-4}
        & \multirow{3}{*}{Shuffle} & Unplayed Songs & 2\\*
        \cmidrule{3-4}
            & & Fixing the Queue & 4\\*
            \cmidrule{3-4}
            & & Close Enough & 2\\*
        \cmidrule{2-4}
        & \multirow{2}{*}{New Song Buffer} & Forgetting to Add & 3\\*
        \cmidrule{3-4}
            & & Desired & 2\\*
        \cmidrule{2-4}
        & \multirow{2}{*}{Listening as a Trajectory} & Listening Rendered as a Line & 4\\*
        \cmidrule{3-4}
            & & Boundary Songs & 2\\*%* Which they change direction on
    \midrule
\end{longtable}

\subsection{Collection Organisation}%todo add Quotes
All the participants had a specific way that they digitally organised their Spotify music collections. Mainly they could be split into two groups: those who placed all their songs into one singular box (the Liked Songs folder) or those who split their collection over multiple playlists.

Only 1 participant felt like they had a vague understanding of their entire collection [Riya], whilst the other 5 felt like they had quite a good grasp on their collection.

Contradicting this however, 3 of the aforementioned 5, upon exploration of their collection in the software application realised there were forgotten or unfamiliar songs to them in their collection.
%? This implies that their grasp on their collection is not as concrete as they stated.

3 of the 4 participants who utilised multiple playlists to organise their collection also attributed benefits to this:\begin{itemize}
    \item knows where a song would be found in, doesn't have to know the exact location, taking off mental load
    \item muscle memory, familiarity
\end{itemize}
%* Implying that this organisation and seperation into distinct regions is good, but as we will find later this level of organisation requires a lot of effort and once it is lost is difficult to regain
The participant who didn't attribute benfits to the structure of their playlists also had an issue with their playlists converging to the same identity[Riya: "over time, my playlists all sort of converge to the same type of song, though they're not meant to"]

Common across all 6 participants is that their collections were continuously growing over time, with only 1 participant stating that they removed songs[Riya] although rarely. %? and only cuz they didn't really belong in the first place

\subsection{Playlist Management}
The 4 participants who organised their collection using playlists also had common concepts and behaviours.

All 4 participants created their playlists with a distinct identity in mind whether this be for a specific artist[Shruthi], a time period[Shruthi, Vedarth], a genre[Josh, Vedarth], a mood or vibe[Riya, Josh, Vedarth] or an activity like the gym[Vedarth, Casper wanted one].

Many of the participants however noted common cases where they felt friction in interacting with their playlists. 2 participants[Shruthi, Vedarth] felt that it was "too much effort" to create playlists with distinct identities that they felt were missing from their collection as there are "too many songs" and ["it would take too much time"].

%? do I need to move this to a different section?
2 participants also noted that when switching between playlists to create queues, they felt that there was a significant amount of mental effort required that put them off from doing so, even though they mentall felt that they needed to create the queue.

3 of the participants also mentioned that they make playlists very infrequently, with 2 of them only making them for each new time period. This was due to ""[insert quote about how much effort it was]

Something that didn't cause friction was the absolute or fixed ordering of the contents of their songs. 3 participants ordered by time[Vedarth, ] or alphabetical order[Shruthi], all remarking that this consistnent order provided "muscle memory" and aligns+reinforces with their mental model
%? Do I have the last bit here?
3 of the participants also ordered songs in the collection to elicit a specific listening experience. 1 participant[Casper] only listened to albums in their canonical order as they exclusively enjoyed that way of listening to them. The other two participants ordered them so that when the didn't shuffle, they could listen through that experience they set out for themselves.

\subsection{Listening Queues}
There were two modes of creating listening queues for the participants: 5 had a passive mode where they created their queues using the shuffle features and 2 created their queues by manually placing songs in the queue.

4 participants built these queues from an individual playlist[] and 4 built them from their entire collection[Casper, Roberto]. When choosing which playlist to choose from and what song to start listening to, 3 participants said that their choice was affected by a recency bias. They felt that they "had a good chance of forgetting songs that were added to the collection earlier on".

Passive listeners also had issues with the shuffle feature of randomly creating an order. 4 participants felt that they had to fix or guide the queue manually due to the shuffled order creating a listening experience that did not match their expected mental queue.

2 participants[] mentioned that they would keep skipping songs if it did not match their expectations until they reached one that was "close enough"[]. % Do I talk about how this is effort and annoying to do? Can't really because I didn't ask this although I should've, can at least bring it up in the project retrospective

2 participants[Josh, Riya] did state that even if the shuffled queue was giving a song that wasn't what they wanted, they would not change it partially due to "not being bothered" and if the playing song was "close enough" to what they were subconsciously expecting.

2 participants[] also mentioned that whilst they used the shuffle feature frequently, they felt/knew that there were songs that hadn't been played for quite some time as the shuffle simply wasn't playing them. This meant that they weren't fully experiencing their collection.

%todo This is sort of duplicated across the evaluation section as well
\subsubsection{New Song Buffer}
3 participants[] mentioned that when they listen to songs passively and are "less aware of what I'm listening to"[Casper?] that they can forget to add these new songs to their collection. 1 participant[Riya] also mentioned that after adding a new song to their collection they would sometimes remove it on subsequent listens due to realising they didn't actually like it. As such 2 participants[] mentioned that they would like a user-facing buffer region feature where newly listened to songs could be permanently added/removed after a few listens.% This would help with managing their collection

\subsubsection{Listening as a Trajectory}
%todo I can probably put in other stuff here to create a proper narrative flow or do I leave that for the final discussion evaluation part?
4 participants mentioned wording that indicated they understood their listening journey to have a direction which was sometimes reflected in the queue. 2 participants[Riya, Josh] mentioned that a factor deciding how they actively mentally change listening direction is when they are on a boundary region of their collection. "When I listen to this song, it has a sad part that'll make me want to listen to more sad songs instead of the direction the queue is currently heading in".

%% ----------------------------------------------------------------
\section{Application Feedback}
Before, the interviews were aimed at undrestanding how the participants organised their digital music collection and how they build and listen to queues using this collection. Whilst using the applicatio, the interview changed to being more about gaining feedback on the usability of the system and each individual feature implemented. Any features that the participants felt would be useful (after interacting with the application) were explored in detail.

As the participants interacted with the application they were asked for their feedback on the implemented features and the attributes:
%todo add in table but what headings?
%todo Attributes/Metadata
%todo Graph feedback
%% ----------------------------------------------------------------
\subsection{Feedback: Attributes/Metadata}

%% ----------------------------------------------------------------
\subsubsection{Value Accuracy}
2 participants disagreed with some of the attribute values for songs in their collection, noting that it did not align with what they expected.
%! However, a lot of people were also like they agreed with it though

%% ----------------------------------------------------------------
\subsubsection{Attribute Opinions}
%% ----------------------------------------------------------------
\paragraph{Key, Mode, Time Signature}%todo
Some cared, some didn't care.

Overall these are more useful in the calculating similarity and also being able to be toggled off (although the toggle off is probably unnecessary as no-one said they actively didn't want it there)

%% ----------------------------------------------------------------
\paragraph{Attribute Rankings: Time above all else}%todo REWORD
[Vedarth] noted that the time axis was easier to understand and more interested in as time/history is more familiar to them. [Casper] also noted that a sped up line of their audio history would be cool, also implying that time was a instinctively useful attribute

\paragraph{Attribute Combinations: Overwhelmed with Choice}%todo REWORD
Many participants[] noted that though specific combinations were interesting they were overwhelemed with trying to find specific combinations and weren't sure where to start or what to do.
X participants agreed that they would like \textbf{customisable presets}, where they could be given combinations to look at that provide easily digestible insights. Using these combinations as a base, the users' could then explore to find their own preferred combinations.

\paragraph{Distributions: Liked seeing ways of representing their Music Taste}%TODO

\paragraph{Desired Feature: Ridge Plot of Histograms for Individual Histograms}%TODO

\paragraph{Attribute: Extreme Ends}%todo FIX
Liked seeing the extreme ends for each attribute (i.e. top 5) although this could be due to the fact that the table view made it easy to see this.
1 participant[Casper] noted that these extreme ends would be useful for automatically creating high danceability playlists for example. There is alreay projects that can do this, though it does beg the question, if we start combing attributes, what sort of playlists would be created. %* However this is something that would of only really been able to check by using the dynamic graph and being able to toggle which attributes are affecting the graph.

%% ----------------------------------------------------------------
\subsection{Feedback: Graph Model}
TODO NEXT

\subsubsection{Graph-Based Suggestions}
3 participants[] noted that they would appreciate seeing song suggestions that clearly show how they slot into their entire collection. Both the ridge plot and dynamic graph would be good for this as they both show the user's entire collection.

\subsubsection{}{Graph Navigation Controls}
Participants noted that the controls were good for navigating in the 2D graph. However, some particiapants[] felt that they would also like a 1st person style of controls, where they could fly through the space of songs. something akin to Minecraft's Creative Mode Controls
%todo maybe put an image of what keys correspond to what for this new controls system

\subsubsection{Song Identification}
Due to time constraints, setting the colour of a song sphere was unfinished. To understand what would be the most preferable dimension to distinguish songs, participants were asked during the study, with the below responses:\begin{itemize}
    \item by colour of\begin{itemize}
        \item Artist = 1[]
        \item Genre = 4[]
        \item Mood/Vibe = 3[]
        \item Discrete Metrics = 0, as participants felt that they would need to see it implemented to see if it would be useful
    \end{itemize}
    \item the image of the album the track belongs to (but only if there was enough visual space to render it)
\end{itemize}

\subsubsection{Dynamic Graph}
All 6 participants[] noted that they would've liked to see their collection using the dynamic graph feature.

All 4 of the participants[] who organised their collection using playlists also expressed interest in seeing how the potential clusters formed in the dynamic graph would map to their created playlists.

\paragraph{Songs as Listening Focus Points}
2 participants[] also noted that when they're listening they would like to be able to select songs as points to listen around (for queue generation and modification)