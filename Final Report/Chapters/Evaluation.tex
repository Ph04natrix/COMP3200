%% ----------------------------------------------------------------
%%? 1st Draft Completed
%% ---------------------------------------------------------------- 
\chapter{Evaluation (2800 words)}
This project aimed to ask two research questions:\begin{itemize}
    \item \textit{how can the organisation of digital music collections be improved to better reflect people's mental models of their collections, without increasing the mental load required?}
    \item \textit{How can we improve the process of creating and controlling}
\end{itemize}

People have a expected mental audio journey. Passive listeners usually have much larger audio journeys that they'd be happy with

%% ---------------------------------------------------------------- 
% Active is blue, Passive is yellow (I don't know why)
\section{Active vs Passive Listening Spectrum}
%% ---------------------------------------------------------------- 
When someone is listening to a sequence of songs they exist somewhere on the passive-active spectrum. This spectrum is concerned with the number of possible song sequences that the person is satisfied listening to.

At the extreme end, a fully passive listener is satisfied with any sequence of any songs, no matter how similar or dissimilar the songs in the listening journey are.

At the other extreme end is a fully active listener, someone who has an exact set of songs that they want to listen to, in an exact order. This order may be for many reasons.

All the participants were somewhere on this spectrum when they were listening to their collection, some would stay in one place or would move about on the spectrum.
[-] Casper usually passive, but then switches to active to make sure that they listen to album in the correct original order
[-] Vedarth passive when doing more mindless activities, but when more focused, preferred to actively create the queue.

%todo draw this spectrum
The interactions made by users (regarding song queues) can be mapped to relative positions on this spectrum:
[-] 100\% Active -> Manually find song then click add to queue
[-] -> Shuffle a small playlist / Song radio
[-] -> Shuffle a large playlist

Keeping track of the next ~10 upcoming songs -> slightly active
Reorder queue -> active
Switch to new playlists -> temporary active then back to passive

\subsection{What is a Listening Journey?}
%% ---------------------------------------------------------------- 
%%* Listening Journeys
%% ---------------------------------------------------------------- 
A listening journey is simply a sequence of songs and as with any sequence, it has the following properties:\begin{itemize}
    \item Initial Item (First Song)
    \item Previous Item (Previously Played Song)
    \item Current Item (Currently Playing Song)
    \item Subsequent item (Next Song)
    \item End Point (Last Song)
    \item Length (Number of Songs Listened to)
\end{itemize}

However this sequence is not simply a sequence of scalar items, but should be thought of as a sequence of vectors, items with direction. As such the overall song queue can be thought of being a listening journey, with an overall direction (and a direction from song to song).

This direction is a vector in an n-dimensional space, where n is the total attributes and metadata for a song. This project only looks at a subset of these attributes and metadata -> the Echo Nest attributes and the Spotify metadata.

What we propose is that the full listening history of a user is comprised of these listening journeys.
However, due to the continuous nature of listening to music in the digital era, further research will have to be done to see if people still have 'end points' or final songs as this was not investigated during the participant study.

A listening journey has 3 parts: its history, its current position, its future:\begin{itemize}
    \item \textbf{History:} a list of songs ordered by when they were listened to (with how much of each song was listened to)
    \item \textbf{Current Position:} the current position in the currently playing track
    \item \textbf{Future:} the trajectory that the listener will follow, comprised of segments that either have a fixed or unfixed/random order:\begin{itemize}
        \item \textbf{Fixed} similar to the history this is a collection of songs which are played through in order
        \item \textbf{Unfixed} the next song to be played is randomly chosen from a collection of 1 or more songs
        \item \textbf{Trajectory} this is the n-dimensional vector between the next song and the current song
    \end{itemize}
\end{itemize}

Both viewing history and the currently playing track are easily viewable and interactable in Spotify (and other streaming services). Spotify does allow for creating and interacting with the future aspect of queues, both fixed and unfixed, however the process to do so still has some friction and can only be done over the entire queue:\begin{itemize}
    \item Fixed Order Queue \(\to\) the user must manually locate and add songs one by one to create a fixed order queue% simple process but can be tedious having to locate each song, especially if they are not stored together -> Vedarth + Riya
    \item Fixed Order Segment \(\to\) a user can reorder songs in the queue to ensure that those songs are played in a fixed order% very easy to do
    \item Unfixed Order Queue \(\to\) a user can click shuffle play to create a new queue of all the songs in a playlist, which have been shuffled to be in a random order. Also they can shuffle the queue once it exists to randomise the order of songs within it. % also very easy
    \item Unfixed Order Segment \(\to\) should a user want to listen to 5 songs in a specific order, then listen to 15 different songs in a random order, then another five songs in a fixed order, there is no built-in way to do this.\begin{itemize}
        \item First the user would have to wait until they've listened through the first 5 fixed order songs.
        \item Then they would have to remember the 5 fixed-order songs they want to listen to at the end and remove those songs from the queue
        \item They could then shuffle the remaining 15 songs to achieve an unfixed/random order.
        \item If they are happy with the shuffle then they can re-add the final 5 fixed-order songs.
        \item However, if at any point they want to reshuffle the queue (but only the random-order songs) they have to go through the entire process again.
    \end{itemize}
\end{itemize}

As can be seen above, whilst Spotify does allow for creating both fixed and unfixed queues, they do not have a way of easily creating unfixed sections of the queue (without losing any desired fixed sections). This project aimed to provide a way for accomplishing this by using the graph visualisations as a base.

Unfortunately, as explained in the Project Retrospective chapter, this feature was not implemented and as such could not be fully tested. However, many participants noted that they would like to see this feature added to Spotify so that they could create song queues with both fixed-order and unfixed-order sections in their queue.
%!-- Now what?? Should this have been in the introduction?

\subsubsection{Trajectory of a Listening Journey}
A listening journey can also be thought of in terms of its trajectory, both between 2 songs, and over multiple songs. These trajectories are comprised of n-dimensional vectors, where n is the sum of all attributes and metadata on a song.

This project aimed at visualising these trajectories by rendering the listening history and upcoming queue as a line in the dynamic graph view, thereby representing all the dimensions of the songs.

Unfortunately, this feature was not completed (as explained in the Project Retrospective Chapter) but when asked to the participants, they all expressed interest in seeing their past and future listening rendered as a directional line. As such this is a concept worth researching further into, to determine how useful it can be, as this was not able to be fully tested during this project.

\subsection{Song Sources for Building Queues}%TODO
When passively listening, listeners often have very little mental energy they want to allocate to controlling the music they listen to[Shruthi Quote]. The easiest way to do this is to pick a playlist and start listening within it.

As such, playlists are an effective solution at reproducing listening queues. For some, one large playlist is enough (often Spotify's Liked Songs) as the listener is happy to listen to any possible sequence of songs in that large playlist. However, for others, they do not want to listen to all their songs at once[Vedarth: larger playlists lose their shuffleability], so they decompose their collection into multiple distinct playlists.

These playlists contain songs which all have a common aspect, which forms the identity of the playlist. This can be anything from having a specific artist or genre in common, as well as being for a specific activity like a workout.

However, due to the enclosed nature of the playlists it is difficult to know how much they overlap and what the true full collection looks like unless they are combined and rendered as one.

Unfortunately, combining multiple playlists into one was a feature that was not implemented, but is worth investigating further. Many participants noted that they wanted to see how similar their playlists were to each other. Further research will be needed to analyse the behaviour of choosing what songs go into which song queues[Vedarth agreed with this]. This will be detailed further in the Future Work Section.

%% --------------
%% Difficulty in maintaining Playlists
%% --------------
\subsubsection{Difficulties in Maintaining Playlists}
%todo fix this paragraph
At first glance, these playlists seem like the perfect solution for easily recreating queues. However, maintaining these playlists can be difficult for some[Riya: playlists started to converge] and applying clean-up is usually avoided [Riya: can't be bothered to go through and remove songs] as listeners just want to listen to their music usually. They are not usually in the mood to perform spring cleaning on their collection.

There is also a perceived heavy mental load associated with the process of creating playlists and adding songs to them. [Vedarth: likes the design and identity of a playlist and pulling different songs together] This puts off user's from creating new playlists with new identities with songs from their collection.[Shruthi+Vedarth would like maybe 2010's playlists but cba to make it]

There is currently too much friction associated with creating playlists, even though there is a desire.

%todo find source to cite how iTunes song tags are good.
\paragraph{Automatic Generation of Playlists}
One possible solution that was proposed in this project, but not attempted due to being low priority was song tags. These are words or phrases that can be attached to any song (similar to the genre metadata) and were a core feature of Apple's iTunes. These song tags can then be used to automatically generate playlists for that tag, taking away the tedious labour from a user. This will be explained futher in the Future Work Section.

\paragraph{Buffer Zone}
Another issue with utilising playlists effectively is adding the right songs when they're found by the user and removing ones that don't belong anymore. Most of the participants didn't remove songs, but that was also because they were more certain that a song belong to the playlist. For the participant who was less certain if a song belonged then they would add it and remove it afterwards if they deemed it necessary. A few participants also noted that they didn't always remember to add songs that they liked listening to, so they would prefer to 

\section{Design Improvements}
%todo how the feedback adds and modifies the original functional requirements
%% ----------------------------------------------------------------
\subsection{Visualising using Continuous Attributes in a Collection of Songs}
The Echo Nest attributes were useful in helping provide more ways for listeners to understand their music collection. Each combination of attributes and songs has a specific visualisation that was found to be the best by the users.

Note: in the below table, Each Song is referring to how a song compares to the rest of the songs in the collection it belongs to.

\begin{longtable}[c]{| c | c | c | c | c|}
    \caption{Optimal views for 1 or more songs and 1 or more continuous attributes} \\
    \toprule
    & \textbf{Singular Song}
    & \textbf{Each Song} % in a playlist/ for a high number of songs
    & \textbf{Overall Distribution}
    & \textbf{Extremities} %Both min and max
    \\ \midrule \endhead

    \textbf{1 Attribute} & \multirow{3}{*}{Table} & Table/1D Graph & Histogram & Table \\*
    \cmidrule{1-1}\cmidrule{3-5}
    \textbf{2 Attributes} & & \multicolumn{3}{c|}{2D Static Graphs} \\*
    \cmidrule{1-1}\cmidrule{3-5}
    \textbf{3 Attributes} & & \multicolumn{3}{c|}{3D Static Graphs} \\*
    \midrule
    \multirow{2}{*}{\textbf{\(>\)3 Attributes}} & Polar Chart/Table/ & \multirow{2}{*}{Dynamic Graph} & \multirow{2}{*}{Ridge Plot} & Table/\\*
    & Line over Ridge Plot & & & Dynamic Graph \\*
    \midrule
\end{longtable}

\subsubsection{Understanding a single song using any and all attributes and metadata} No novel visualisations were required. Of note, some participants[Riya, ] liked the polar chart as a way of rendering all the continuous attributes.

\subsubsection{Understanding multiple songs according to 1 attribute}
Conventional methods sufficed: tables, histograms, bar charts, etc.

\subsubsection{Understanding multiple songs according to 2 continuous attributes} The conventional method of 2-dimensional cartesian graphs were sufficient for the participants. It allowed for viewing the collection as a whole and for also looking at and comparing individual songs. Finding outliers and max and minimum points was a trivial process in this view.

\subsubsection{Understanding multiple songs according to 3 continuous attributes} The 3-dimensional cartesian graph was useful here but harder to navigate and represent on a 2D desktop screen. This would be easier to interact with using a 3D medium, such as AR or VR.

\subsubsection{Understanding multiple songs according to more than 3 attributes} The conventional cartesian graphs do not hold up anymore. Although it was not developed in time to be fully evaluated, the design of the dynamic graph was evaluated by the participants. Their feedback indicated that this view is an effective method of better understanding their collection using more than 3 attributes. 2 participants in particular were particularly excited as they felt it better matched their mental model of their music collection and the way they listen.

Whilst the dynamic graph is good at visualising multiple songs using multiple attributes, it collapses the attributes in order to do so. This makes it harder to distinguish how each attribute is affecting the collection. To see how the collection as a whole can be represented by the attributes, the best visualisation is a ridge plot \(\to\) histograms of each continuous attribute layered over each other. This feature was designed but not implemented, so the design was evaluated, but requires further research.

The participants all found the ridge plot as a significantly useful feature. 1 participant[Casper] noticed that it was very effective as mini-map of the whole collection. When combine with the range sliders for filtering, this feature had more application. 1 participant[Casper] stated that they would prefer to use this filterable ridge plot to filter out the table and graph views over the conventional filtering method.

\subsection{Attribute Combinations: Overwhelming Choice}
All the participants enjoyed seeing their collections according to different combinations of attributes. However, with there being 12 attributes in total that could be applied to the axis there was a significant amount of combinations to go through:\begin{itemize}
    \item 66 combinations over 2 axes
    \item 220 combinations over 3 axes
\end{itemize}
This is such high number of combinations that most people would not bother trying to go through all of them. The patricipants only went through 3 or 4. To assist with understanding and using the static cartesian graphs, 2 participants noted that they would like suggested combinations to start with. This is detailed more in the Future Work section.

\subsection{The Echo Nest Attributes}%TODO Finish this section properly
Users enjoyed being able to see their collection defined in terms of the Echo Nest attributes as some data points affirmed pre-existing thoughts about their songs.

It was also an effective way of defining one's music tastes.[who?]

However, not all the attributes were useful to everyone, most notably key, mode and time signature. Further testing on a higher sample size would be required to fully test which attributes are more relevant over others.

The most influential dimension turned out to be time as this was easiest to interpret from the user's side as noted by one participant[Vedarth quote]. [Casper] also noted that a sped up line of their audio history would be cool, also implying that time was a instinctively useful attribute

The attributes also required some explanation as they were not all self-explanatory. As such for these to be introduced in a more user-facing role and to be rolled out to the public, a walkthrough would need to be created, going through each attribute and giving a brief understandable description.

%? Put in table of what attributes were the best and maybe what combos of attributes were good?

\subsection{Distinguishing Songs in the Graph Views}
The static graph views were highly effective at understanding one's entire music collection and learning the different landscapes of their collection. However, a major trade-off with the graph views (including dynamic) is the loss of individual song identifiability. 

To ensure the individual songs were still identifiable in the graph views, the following options were queried with the participants to receive feedback:\begin{itemize}
    \item \textbf{Title Text:} Most recognisable and interpretable. Usually not enough space to full render however.
    \item \textbf{Album Image Square:} Less recognisable than song title, but still recognisable. Requires much less space to fully render.
    \item \textbf{Coloured Sphere:} Least recognisable but requires the least amount of space to render.
\end{itemize}

The approach taken in the project was to render the songs as 3D spheres, with planned functionality to colour them based on a discrete attribute or metadata. 5 participants[no Josh] expressed that they would want the colour to be determined by genre and 3 participants[] stated they would want to colour by the mood/vibe of the song.

One participant noted that they would want to colour by artist, however this is likely due to the fact that they build their collection around a specific set of a few artists.

The right method of distinguishing songs can also have a secondary bonus effect, allowing for one to roughly see the distribution of a discrete attribute/metadata on their collection.

5 participants[not Riya] said they would prefer album images over 3D spheres as this method hit the right balance of not taking up too much visual space (resulting in overlap and obscurity) and still alowing for song recognition. However, they all agreed that they did not mind the song spheres as well and would only prefer the album images to render when zoomed in enough to be legibile.

\subsection{Graph-Based Suggestions}
Currently in the Spotify app, suggestions are just limited to a small list of five after the end of a playlist. 3 participants[] noted that they would heavily appreciate seeing suggested songs rendered in their collection, so they can see how it fits into their existing landscape.

\subsection{Graph Navigation Controls}
All the participants felt that the controls for the 2D graph were satisfactory, however some[] noted that they would've also preferred an alternate set of controls for the 3D graphs.

Instead of the implemented controls using the mouse and having the graph as the center of the 3D scene, a more first person videogame-esque control system was suggested. This would involve using the keyboard to move the camera in all directions to better explore the 3D space.