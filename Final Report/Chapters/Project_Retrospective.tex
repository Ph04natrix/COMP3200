%% ----------------------------------------------------------------
%%? 1st Draft In Progress
%% ----------------------------------------------------------------
\chapter{Project Planning Retrospective (1056 words)}
%todo fix the intro section
This section will explore in detail the planning and development of the project itself, throughout the stages: research, designing, development, evaluation and the final write-up.

I will go over what were the significant issues with the approach I took. I will then talk about how I would this project were I to start it over again.%todo make this sound better

Since this project had a very limited timeframe, ensuring that progress remained on schedule was critical. %? do I need to talk about how keeping things on schedule was important here?

%% ----------------------------------------------------------------
\section{What Went Well}
%% ----------------------------------------------------------------
One of the main things that significantly helped with planning the project and managing all the things I needed to do was building a hierarchical mindmap in Miro. Miro is a diagramming tool that allows for building any type of diagram. The reason Miro's mindmap was useful was that it allowed for children nodes to be toggled, meaning that they were visually hidden, but still accessible.

This meant that I could decompose the project into its significant core parts. Each section could then be decomposed into its significant parts. This significantly reduced the mental load and allowed for me to better breakdown each aspect of the project as much as I needed to.
\begin{figure}
    \includegraphics[angle=90, scale=0.35]{Miro_Mindmap_Depth_1,2}
    \caption{The Miro Mindmap Tree only showing nodes to a depth of 1 or 2}
\end{figure}

%! Do I talk about how it also didn;t work. Because I could hide stuff it was also a bit easier to forget about (about what???)

%% ----------------------------------------------------------------
\section{Pitfalls}
%% ----------------------------------------------------------------
There were many issues that occurred, some within my control and some due to external sources:

%% ----------------------------------------------------------------
\subsection{Acquiring Song Attributes for all Songs in a User's Library}
%% ----------------------------------------------------------------
[2014] Spotify buys EchoNest - now EchoNest's API is locked behind a Spotify Premium Account (this wasn't an issue due to already having a premium account, but this is a paywall), however this was also a sort of blessing in disguise as it would've meant that I only need to work with one API.
    
[27 Nov 2024] Spotify announces they are deprecating several endpoints for applications made after the 27th November - these endpoints included both \lstinline|Get Track's Audio Features| and \lstinline|Get Track's Audio Analysis| which were key for the project

These audio features (or attributes) were necessary as they would provide the numerical basis for mapping the songs and helping control the audio journeys.
To ensure that this didn't derail the project, after much searching an alternative was found, namely the SoundCharts API. This API had an endpoint, that given a track's Spotify ID, would provide the attribute data that Spotify had deprecated access to.
However, this data was less accurate (only to 2 decimal places) and was also behind a paywall. For one month, the cost of access for 500,000 API calls at a 30\% academic discount amounted to 125 USD. This was within the budget of the project. Initially, the plan was to purchase one month once development had finished. As such the API access would be used only when it was fully needed.

[Late March 2025] Due to significant issues with purchasing the API access using the University's system, another alternative method to gaining the attributes had to be found. This solution was found in Exportify, a web tool built by Pavel Komarov. This tool accesses the Spotify API, including the newly deprecated endpoints, to allow for exporting a spotify user's library to a \lstinline|.csv| file.

This tool was made before the deprecation announcement and is free to use, making it a very suitable replacement. Futhermore, the attribute values are to the original precision as provided by Spotify. The tool also allowed for exporting of individual playlists, making it easier for my software application to know how one's full collection was composed by the playlists and the catch-all liked songs.

Unfortunately, due to the sudden nature of the Spotify deprecation announcement, an alternative had to be chosen very quickly meaning that Exportify wasn't found until very, very late in development. This meant that the API setup portion of the project took longer than it theoretically could've, as the workflow using Exportify's \lstinline|.csv| files is much simpler (as shown in the figure NUMBER below). The ideal method would've been to continue searching for alternatives after finding SoundCharts allowing for Exportify to be found sooner and be integrated into the application state flow from the start, however due to the time constraint in needing to submit a project brief, this was not done.%\includegraphics[]{} %todo Show the setup flow with SoundCharts versus with Exportify

%%! unwritten ----------------------------------------------------------------
\subsection{Unfinished Code in the Development Stage}
%% ----------------------------------------------------------------

%% ----------------------------------------------------------------
\subsection{
    Losing sight of the Goal
    Only learning and understanding the true goal of the project near the end.
}% Lost sight of the goal at the core of this project, which was the dynamic graph
%% ----------------------------------------------------------------

This project whilst in a limited timeframe, was still quite long, hvaing taken the better part of 7 months. During this process the true goal/research questions were only fully understood quite late into the project sadly.

The true/original niche/gap in the research was investigating better ways to create listening journeys, mainly using the graph visualisations as a foundation for controlling and viewing them. This is because, with the advent of the digital streaming era, there has not been any research (none that I could find at least) for creating better tools for users. As discussed in the Evaluation, there is a desire for these tools.

However, during development, as the static graph views were being created, I realised that these were more specifically useful in a data analysis perspective. Unfortunately, data analysis on a user's music library, especially using the EchoNest (now Spotify's) attributes, is something that has been investigated extensively, as mentioned in the background chapter.

In hindsight developing the staic graph views should've been allocated to be done after the dynamic views so that the listening journeys could be accomplished as fast as possible. Static cartesian graphs were intially planned first as I did not realise at the time that they were significantly less useful for interacting with listening journeys than the dynamic view.

One participant in the evaluatory study also mentioned that they liked there being an absolute order to their songs/collection that they could return to. This absolute order synergises better with the dynamic graph, as the static graph produced different distributions for each combination of attributes on the axes.
%This was Vedarth, was there anyone else?

%% ----------------------------------------------------------------
\section{Initial vs Final vs Ideal Project Plan} % ? how much explanation do I need, will Gantt charts be enough
%% ----------------------------------------------------------------
Below are three project plans:\begin{itemize}
    \item \textbf{Initial Plan} \(\to\) this is what was planned initially near the start of the project
    \item \textbf{Actual Progress} \(\to\) this is the actual progress of the project
    \item \textbf{Ideal Plan} \(\to\) upon retrospection, this is how the project should be approached if to be done again
\end{itemize}

%% ----------------------------------------------------------------
\subsection{Initial Plan}
%% ----------------------------------------------------------------

%% ----------------------------------------------------------------
\subsection{Actual Progress}
%% ----------------------------------------------------------------

Key Points:
- reduce the amount of features
    - should've only done the dynamic graph as other people have done the static cartesian graphs before
    - the evaluation itself took ages 
- either of the below:
    - do an initial survey of people before design and development -> get a better sense of requirements
    - had a review meeting with the review team on the application - was supposed to do this but hadn't finished enough code in time and they both left the country early

%% ----------------------------------------------------------------
\subsection{Ideal Plan}
%% ----------------------------------------------------------------