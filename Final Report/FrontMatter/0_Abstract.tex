%% ----------------------------------------------------------------
%%? 187/200 words
%% ----------------------------------------------------------------
Music Streaming Services have made it significantly easier to listen to and collect any amount of songs, leading to much larger song libraries. Whilst there has been heavy research into improving how to recommend songs, there has been a lack of improving the organisation of these large song libraries and better ways to listen across them. 

The only method of organisation is the hierarchical use of playlists, however these have a mental tax attached to the creation and mantainance of them. This project will investigate two different approaches to viewing and organising songs: static cartesian graphs and a dynamic similarity graph. These graphs will then be used as a basis for novel visualisations of a user's listening history and song queue. This project will also investigate how the listening behaviours of Spotify users can be improved with novel software features.

Attributes originally extracted from the Echo Nest (now part of Spotify) will be used to map songs in the static graphs and compute the similarity for the dynamic graph. These features will be implemented using a desktop application (using web technologies) that interacts with the Spotify API.